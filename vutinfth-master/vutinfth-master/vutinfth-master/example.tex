% Copyright (C) 2014-2017 by Thomas Auzinger <thomas@auzinger.name>

\documentclass[draft,final]{vutinfth} % Remove option 'final' to obtain debug information.

% Load packages to allow in- and output of non-ASCII characters.
\usepackage{lmodern}        % Use an extension of the original Computer Modern font to minimize the use of bitmapped letters.
\usepackage[T1]{fontenc}    % Determines font encoding of the output. Font packages have to be included before this line.
\usepackage[utf8]{inputenc} % Determines encoding of the input. All input files have to use UTF8 encoding.

% Extended LaTeX functionality is enables by including packages with \usepackage{...}.
\usepackage{amsmath}    % Extended typesetting of mathematical expression.
\usepackage{amssymb}    % Provides a multitude of mathematical symbols.
\usepackage{mathtools}  % Further extensions of mathematical typesetting.
\usepackage{microtype}  % Small-scale typographic enhancements.
\usepackage[inline]{enumitem} % User control over the layout of lists (itemize, enumerate, description).
\usepackage{multirow}   % Allows table elements to span several rows.
\usepackage{booktabs}   % Improves the typesettings of tables.
\usepackage{subcaption} % Allows the use of subfigures and enables their referencing.
\usepackage[ruled,linesnumbered,algochapter]{algorithm2e} % Enables the writing of pseudo code.
\usepackage[usenames,dvipsnames,table]{xcolor} % Allows the definition and use of colors. This package has to be included before tikz.
\usepackage{nag}       % Issues warnings when best practices in writing LaTeX documents are violated.
\usepackage{todonotes} % Provides tooltip-like todo notes.
\usepackage{hyperref}  % Enables cross linking in the electronic document version. This package has to be included second to last.
\usepackage[acronym,toc]{glossaries} % Enables the generation of glossaries and lists fo acronyms. This package has to be included last.

% Define convenience functions to use the author name and the thesis title in the PDF document properties.
\newcommand{\authorname}{Rebeka Koszticsak} % The author name without titles.
\newcommand{\thesistitle}{Enhancement of Footwear Impressions} % The title of the thesis. The English version should be used, if it exists.

% Set PDF document properties
\hypersetup{
    pdfpagelayout   = TwoPageRight,           % How the document is shown in PDF viewers (optional).
    linkbordercolor = {Melon},                % The color of the borders of boxes around crosslinks (optional).
    pdfauthor       = {\authorname},          % The author's name in the document properties (optional).
    pdftitle        = {\thesistitle},         % The document's title in the document properties (optional).
    pdfsubject      = {Subject},              % The document's subject in the document properties (optional).
    pdfkeywords     = {a, list, of, keywords} % The document's keywords in the document properties (optional).
}

\setpnumwidth{2.5em}        % Avoid overfull hboxes in the table of contents (see memoir manual).
\setsecnumdepth{subsection} % Enumerate subsections.

\nonzeroparskip             % Create space between paragraphs (optional).
\setlength{\parindent}{0pt} % Remove paragraph identation (optional).

\makeindex      % Use an optional index.
\makeglossaries % Use an optional glossary.
%\glstocfalse   % Remove the glossaries from the table of contents.

% Set persons with 4 arguments:
%  {title before name}{name}{title after name}{gender}
%  where both titles are optional (i.e. can be given as empty brackets {}).
\setauthor{}{\authorname}{Bsc}{female}
\setadvisor{Ao.Univ.Prof. Dipl.-Ing. Dr.techn.}{Robert  Sablatnig}{}{male}

% For bachelor and master theses:
\setfirstassistant{Dipl.-Ing.}{Manuel  Keglevic}{}{male}
%\setsecondassistant{Pretitle}{Forename Surname}{Posttitle}{male}
%\setthirdassistant{Pretitle}{Forename Surname}{Posttitle}{male}

% For dissertations:
%\setfirstreviewer{Pretitle}{Forename Surname}{Posttitle}{male}
%\setsecondreviewer{Pretitle}{Forename Surname}{Posttitle}{male}

% For dissertations at the PhD School and optionally for dissertations:
%\setsecondadvisor{Pretitle}{Forename Surname}{Posttitle}{male} % Comment to remove.

% Required data.
\setaddress{Address}
\setregnumber{01325492}
\setdate{04}{01}{2020} % Set date with 3 arguments: {day}{month}{year}.
\settitle{\thesistitle}{Enhancement of Footwear Impressions} % Sets English and German version of the title (both can be English or German). If your title contains commas, enclose it with additional curvy brackets (i.e., {{your title}}) or define it as a macro as done with \thesistitle.
%\setsubtitle{Optional Subtitle of the Thesis}{Optionaler Untertitel der Arbeit} % Sets English and German version of the subtitle (both can be English or German).

% Select the thesis type: bachelor / master / doctor / phd-school.
% Bachelor:
%\setthesis{bachelor}
%
% Master:
\setthesis{master}
\setmasterdegree{dipl.} % dipl. / rer.nat. / rer.soc.oec. / master
%
% Doctor:
%\setthesis{doctor}
%\setdoctordegree{rer.soc.oec.}% rer.nat. / techn. / rer.soc.oec.
%
% Doctor at the PhD School
%\setthesis{phd-school} % Deactivate non-English title pages (see below)

% For bachelor and master:
\setcurriculum{Visual Computing}{Visual Computing} % Sets the English and German name of the curriculum.

% For dissertations at the PhD School:
%\setfirstreviewerdata{Affiliation, Country}
%\setsecondreviewerdata{Affiliation, Country}


\begin{document}

\frontmatter % Switches to roman numbering.
% The structure of the thesis has to conform to
%  http://www.informatik.tuwien.ac.at/dekanat

\addtitlepage{naustrian} % German title page (not for dissertations at the PhD School).
\addtitlepage{english} % English title page.
\addstatementpage

\begin{danksagung*}
\todo{Ihr Text hier.}
\end{danksagung*}

\begin{acknowledgements*}
\todo{Enter your text here.}
\end{acknowledgements*}

\begin{kurzfassung}
\todo{Ihr Text hier.}
\end{kurzfassung}

\begin{abstract}
\par
Shoeprint images are one of the most commonly secured evidences on crimescenes.
Even though automatic shoeprint processing is a highly researched topic, the final identifacion is usually done by human forensic experts.
The two main steps of shoeprint identification are enhancement and matching. 
\par
In this thesis the possibilities for enhancement of shoeprint samples from a real-life dataset are investigated.
The main challange of this task is to correctly filter the pattern regardless the versitile, possibly heavily structured and clutterd noise on the samples.
Two main approaches are examined, pattern enhancement and noise suppression.
Among fully automated methods, a semi-automated technique is also tested, where user input is required for noise separation.
\par
The main goal of this work is to find a universal approach which is able to filter and enhance the shoeprint data even in the presence of noise and the possible low image quality.
Based on the experiences acquired while investigating the possible techniques a new noise-supression pipeline for shoeprint images is introduced.
The noisy pixels are identified based on the Fourier-Mellin features of their multi-sized neighborhood.
In the same time a model is built about the average appearance of noise, to eliminate that structure from the foreground as well.
Additionally a gradient based line detector is also applied and the edge structures of the shoeprint are clustered to distinguish between pattern and noise edges.
The experimental results show that the processed images are clearer, the pattern is sharper whereas the noise is either completely eliminated in the background or suppressed in the foreground.
Furthermore based on the results of three differenet basic image descriptor features, the enhanced shoeprints have higher matching rate to their ground-thruth samples than the original images.


\end{abstract}

% Select the language of the thesis, e.g., english or naustrian.
\selectlanguage{english}

% Add a table of contents (toc).
\tableofcontents % Starred version, i.e., \tableofcontents*, removes the self-entry.

% Switch to arabic numbering and start the enumeration of chapters in the table of content.
\mainmatter

\chapter{Introduction}

\par
Shoeprints found on crimescenes can be important hints or evidences in a criminal investigation \cite{kong2014novel}.
Event though on one thrid \cite{alexandre1996computerized} of crimescenes usable shoepatterns can be secured, there is no fully automatized algorithm available yet, which is able to identify and match those prints with the original shoe sole.
Because of that human power is needed \cite{wang2014automatic} to recognize and analyze the found patterns.
The work of forensic experts is not only time consuming and expensive, there is no guarantee about the objectivness of the final outcome\cite{gueham2008automatic}, furthermore the stages of the human matching process are unclear and not necessarily reproducible.
\par
There is an excessive amount of research already done \cite{rida2019forensic} in order to help or repleace the work of forensic experst.
There is however no algorithm published yet, which can be relaiably used in varying conditions and sample quality.
One reason for that are the already mentioned versitile conditions, the features and properties of the pattern on the shoe, like age, material, etc., the characteristics of the ground where the shoeprint is left and enviromental conditions like for example the weather highly influence the overall quality of the acquired sample.
Those high amount of factors result in changing appearance of the prints of the same shoe causing high intra class variance while clustering.
Additionally there is a lack of universal, wide ranged database \cite{rida2019forensic} which correctly depicts the common scenarios occuring on real-life crime scenes.
\par
In 2014 a new database, called FID-300 \cite{kortylewski2014unsupervised} was released which aims to solve the database problem described above.
It contains over 1000 reference shoeprint patterns acquired in a laboratory.
Moreover the database introduces 300 new shoeprint samples collected by the police providing an insight on images forensic experts are working on the daily basis.

\section{Problem Definition}
\par
There are two main stages of automatic shoeprint identification, filtering, where the shoeprint pattern is separated from background and enhanced as well, and matching where the corresponding shoe is determined.
Instead of automatizing the entire shoeprint recogition pipeline this work only focuses on the possible ways of icreasing the sample quality.
Because of the mentioned absence of general, appropiate database it is difficult to compare the already available methods.
Furthermore it is also challanging to estimate which one is applicable in a real-life scenario.
In this thesis multiple possible enhancing techniques are developed and tested in order to find a method which is able to cope with samples taken from real crime scenes.
\par
For evaluation and testing the FID-300 database is used.
The dataset contains both in a laboratory acquired as well as on a crime secured shoeprint patterns.
The goal of this work is to define an image processing pipeline which is able to correctly identify and enhance the shoepatterns and eliminate or suppress the noise on the pattern samples regardless the quality of the image.
A secondary objective is to gain an overview about the algorithms already published, and make an estimation which methods are applicable in real-life scenarios based on their performance on the FID-300 database.

\section{Challanges}
\par
There are two main obsticles in the topic of shoeprint enhancement and in automatic shoeprint matching in general, the versitile image quality and appearance and the lack of universal and wide database.
The shoeprint patterns are varying, there are approaches available which build models for given structures of the shoeprint \cite{tang2010footwear}, \cite{alizadeh2017automatic}, but no detailed, uniform representation for the entire shoeprint is possible.
Moreover there is a high inference of noise from multiple sources.
The ground where the shouprint is found is considered as noise expect in the rear case when it is left on a non changing, even surface.
The produced print of the same shoepattern varys on different type of surface.
Additionally the roughness and unevenness of a given type of surface also distorts the original pattern.
Furthermore other objects on the ground, on or behind the left shoeprint can cover or distort the original pattern, or they can prevent to leave a print on their area completely.
Besides that the pattern on the original shoe can also be distorted or modified compared to the new version.
Distortions caused by usage are valuable information about the owner, on the other hand they make it more difficult to match the pattern with their unused pairs.
Additional objects between the structures of the shoeprint also alter the original appearance.
Lastly, there are multiple shoeprint securing methods producing different results for the same print \cite{katireddy2017novel}. 
The shoeprint lifting technique used depends on the properties of the ground. 
Those two factors, the securing method and the floor, also determine if the positive or the negative, the actual pattern or the space between the shoeprint structures, image is captured.
\par
The non-existing universal database causes that two published methods are difficult to compare based on their results since they are using different testing images.
The used dataset is not necessarily published \cite{katireddy2017novel}, \cite{dardi2009texture} making it impossible to reproduce the reulst in those cases.
Additionally the handcrafted databases can be biased, and allow such restrictions and modifications which do not correlate with real-life scenarios \cite{rida2019forensic}.
The used samples are either synthetically generated and computationally distorted \cite{de2005automated}, \cite{gueham2008automatic} or exlude low quality and noisy images \cite{dardi2009texture}, \cite{tang2010footwear}.
Because of that it is difficult to compare their performance and to estimate which one of the published approaches are applicable on the FID-300 database.
Furthermore it is challanging to plan a new algorithm based on the published results because their lack of a uniform baseline.

\section{Contribution}

\section{Structure of the Work}

\chapter{Additional Chapter}
\todo{Enter your text here.}

% Remove following line for the final thesis.
\input{intro.tex} % A short introduction to LaTeX.

\backmatter

% Use an optional list of figures.
\listoffigures % Starred version, i.e., \listoffigures*, removes the toc entry.

% Use an optional list of tables.
\cleardoublepage % Start list of tables on the next empty right hand page.
\listoftables % Starred version, i.e., \listoftables*, removes the toc entry.

% Use an optional list of alogrithms.
\listofalgorithms
\addcontentsline{toc}{chapter}{List of Algorithms}

% Add an index.
\printindex

% Add a glossary.
\printglossaries

% Add a bibliography.
\bibliographystyle{alpha}
\bibliography{intro}

\end{document}